\documentclass[presentation]{beamer}
%\documentclass[handout,notes=show]{beamer}
\usepackage{mystyle}

% for handout with 8 slides on one page uncomment
%\usepackage{pgfpages}
%\pgfpagesuselayout{8 on 1}[a4paper,border shrink=2mm] 

\setbeamertemplate{note page}[plain]

\title{Title}

\author{Mirjam Friesen}
\date{\today}

\begin{document}

% Title
% use the standart title page
\frame{\titlepage}
% or style it on your own
\begin{frame}[plain]
    \centering
    \begin{beamercolorbox}[sep=8pt,center]{title}
    \usebeamerfont{title}
    \inserttitle
    \end{beamercolorbox}
    \begin{columns}
    \column{0.4\textwidth}
    \begin{figure}
    \vspace{2cm}
    First figure
    \vspace{3cm}    
    \end{figure}
    \column{0.5\textwidth}
    \begin{figure}
    \vspace{1cm}
    Second figure
    \end{figure}
    \end{columns}
    \begin{columns}
    \column{0.4\textwidth}
    \centering
    {\usebeamerfont{author}
    \insertauthor
    }
    \\ \medskip
    {\usebeamerfont{date}
    \insertdate
    }
    \column{0.6\textwidth}
    \end{columns}
\end{frame}
\note{}
\setcounter{framenumber}{0}
%------------
\begin{frame}{A Theorem}
    \begin{theorem}[with Coauthor]
        This astonishing thing is true.
    \vspace{2ex}
    \begin{itemize}
    \setlength\itemsep{1.5ex} % some times is a good idea so change the spacing
    \item first
    \item second
    \item third
    \end{itemize}
    \end{theorem}
\end{frame}
\note{Tell people about cool result}
%---------------
\begin{frame}{Frame with a Proposition}
    \begin{proposition}[David Dundas\cite{Motor}]
        Jeans on.
    \end{proposition}
\end{frame}
\note{}
%------------
\begin{frame}{Demonstration of a block}
    \begin{block}{Colorful inequalitity}
    \begin{align*}
     y &\leq 2 
       - {\color{C0}\omega^a(\delta^{in}(c))} 
       - {\color{C2}\omega^c(\delta^{in}(a))} 
       - {\color{C1}\omega^b(\delta^{in}(d))} 
       - {\color{C3}\omega^d(\delta^{in}(b))}\\
     y &\leq 2 - \omega^a(\delta^{in}(d)) - \omega^d(\delta^{in}(a)) 
       -\omega^b(\delta^{in}(c)) - \omega^c(\delta^{in}(b))
    \end{align*}
    \end{block}
\end{frame}
\note{}
%------------
\appendix 
\begin{frame}{Bibliography}
	\tiny
    \bibliographystyle{plain}
    \bibliography{references}
\end{frame}
\note{}
\addtocounter{framenumber}{-1}
%------------
\end{document}
